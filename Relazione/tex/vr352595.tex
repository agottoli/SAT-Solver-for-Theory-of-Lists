% -*- coding: utf-8 -*-

% Relazione implementazione Algoritmo di Chiusura di Congruenza
% per SODDISFACIBILITA' della Teoria delle Liste

\documentclass[a4paper,11pt]{article} %{report} %{article}
\usepackage[italian]{babel} % per scrivere in italiano
\usepackage[utf8]{inputenc} % per usare lettere accentate direttamente ne .tex
\usepackage{amssymb, amsmath} % per simboli e ambienti matematici
%\usepackage{fourier} % per font utopia
\usepackage{hyperref}
\usepackage{fancyhdr}
\pagestyle{plain}

\newcommand{\campo}{\texttt}

%\hyphenation{Ackerman} % per non sillabare la parola

%\setcounter{secnumdepth}{-2}
\setcounter{tocdepth}{2} % per inserire nell'indice fino a 2=subsection

%\evensidemargin=1cm 
\oddsidemargin=1.3cm
\headsep = 0cm
\textheight = 640pt
\textwidth = 380pt

\begin{document}

\title{Implementazione della procedura di soddisfacibilit\`a per la Teoria delle Liste}
\author{Alessandro Gottoli vr352595}
%\date{}
\maketitle
%\tableofcontents % genero l'indice (doppia compilazione)

\section*{Introduzione}
La procedura di soddisfacibilit\`a per la teoria delle liste
di questo progetto \`e stato implementato nel linguaggio di 
programmazione Java per essere eseguibile su diversi sistemi
operativi ed \`e stato testato su Windows e Linux.
Il progetto si divide in tre: interfaccia, parser, chiusura di congruenza.
%\vspace{-1ex}
%\begin{itemize}
%\item \hyperref[sec: interfaccia]{Interfaccia}
%\vspace{-1ex}
%\item \hyperref[sec: parser]{Parser}
%\vspace{-1ex}
%\item \hyperref[sec: algo cc]{Algoritmo di chiusura di congruenza}
%\end{itemize}
\vspace{-1ex}
\section{Interfaccia}\label{sec: interfaccia}
L'inserimento della formula avviene tramite un interfaccia grafica e si pu\`o
digitare direttamente nell'area di testo oppure aprire comodamente da file. La
stringa contenuta nel file viene visualizzata, dando modo all'utente di modificarla
e salvarla in un altro file (con ``alert'' in caso si tentasse di sovrascrivere un file esistente).
Per\`o in formule molto lunghe questa operazione appesantirebbe troppo l'interfaccia, 
quindi se la dimensione \`e maggiore di 50000 caratteri, si indica solamente il percorso del file
caricato.
Tutti i pulsanti vengono abilitati solo all'occorrenza, limitando al minimo i possibili errori degli utenti.
Per esempio; %se non viene digitata o 
%caricata nessuna formula, 
il pulsante ``Start'' %che avvia la procedura 
che avvia l'analisi non \`e attivo se la formula \`e vuota; 
oppure finch\'e l'algoritmo \`e in esecuzione, la barra %che consente di cambiare tipo di algoritmo
dei menu
%e operazioni sui file 
risulta non attive e l'area di testo non modificabile, 
costringendo l'utente ad attendere la fine dell'algoritmo 
o a terminarla con il pulsante ``Abort'' che appare quando inizia l'algoritmo.

Per apprezzare %appieno 
l'interfaccia grafica occorre utilizzare formule corpose, cos\`i
si noter\`a %in 
che fase dell'algoritmo % ci troviamo. 
\`e in esecuzione.
%Infatti per prima cosa si visualizzer\`a la
%scritta che informa che si sta facendo il parsing e la costruzione del grafo con 
Durante la fase di parsing verr\`a visualizzata
una
barra di caricamento che indica la progressione con la percentuale elaborata. 
%Quando questa fase sar\`a completata
%sar\`a segnalato indicando anche il tempo impiegato e apparir\`a l'informazione che si 
%sta eseguendo 
Per l'algoritmo di chiusura di congruenza %; e siccome in questa fase 
non è 
possibile stimare %a priori 
il tempo necessario, %si indica semplicemente che il 
%programma sta elaborando con 
quindi si usa una barra ``indeterminata'' animata per indicare che il programma sta elaborando.
%Qualora la computazione risultasse troppo lunga \`e possibile terminarla premendo il tasto
%``Abort'' che apparir\`a al posto dello ``Start''.
%
%Senza dilungarci troppo, lasciamo l'interfaccia si presenti da sola quando si testeranno
%le formule di esempio allegate assieme all'implementazione.
Al completamento di ogni fase verr\`a visualizzato il tempo impiegato.
Passiamo alle fasi pi\`u importanti.  

\vspace{-1ex}
\section{Parser}\label{sec: parser}
Una volta inserita la formula della quale vogliamo calcolare la soddisfacibilit\`a,
questa stringa viene passata ad un parser che la analizza carattere per carattere effettuando
contemporaneamente analisi sintattica e costruzione del grafo.
\vspace{-1ex}
\subsection{Analisi sintattica}
% = != !atom(not atom)
L'analisi sintattica controlla che la formula sia stata scritta rispettando le seguenti 
regole sintattiche: 
\vspace{-1ex}
\begin{description}
\item[elemento:] nome senza spazi. Esempio: elem.
\vspace{-1.5ex}
\item[funzione:] il nome rispetta la regola per l'elemento, 
	i parametri sono a loro volta funzioni o elementi e sono racchiusi tra parentesi tonde `(' e `)'
	e separati %da virgole 
	`,' o %punti e virgole 
	`;'. Esempio: fun(el, g(a)).
\vspace{-1.5ex}
\item[uguaglianze/disuguaglianze:] sono costituite da funzioni o elementi separati da `='/`!='. Esempio: c = fun(b), a != fun(b).
%\vspace{-1ex}
%\item[disuguaglianze:] sono come le uguaglianze ma si usa i caratteri `!='.% Esempio: elem != fun(b).
\vspace{-1.5ex}
\item[atom:] viene trattato come una funzione con un solo parametro e per indicare la sua negazione
	si fa precedere da `!'.% Esempio: atom(c). %
\vspace{-1.5ex}
\item[clausole:] sono uguaglianze, disuguaglianze, atom o !atom separati da virgole `,' 
	o punti e virgole `;'. Esempio: a = f(b); !atom(g(a,b)).% , el != g(a) , atom(a)
\end{description}
\vspace{-1ex}
Gli spazi vengono tutti ignorati tranne quelli presenti nel nome di una funzione o di un elemento
(che restituiscono errore sintattico) per dare pi\`u flessibilit\`a all'utente di scrivere 
le formule nel formato che preferisce.

%L'analisi semantica invece controlla che gli elementi e le formule abbiano
%sempre la stessa ariet\`a. Questo ci garantisce che lo stesso nome non possa essere usato sia
%per funzioni con un numero diverso di argomenti sia per indicare un elemento.
%Per esempio la formula ``f(a,b) = f'' restituir\`a errore sintattico perch\'e \`e scritta nella
%sintassi giusta, ma ``f'' viene usata nel modo sbagliato.

\vspace{-1ex}
\subsection{Costruzione del grafo}
La costruzione del grafo istanzia degli oggetti ``nodi'' ed ognuno rappresenta una funzione
o un elemento riconosciuto durante il parsing. Siccome nella stringa un elemento pu\`o
comparire pi\`u di una volta, \`e importante non duplicare il nodo corrispondente a quell'elemento.
Controllare %ogni volta 
se un nodo %\`e gi\`a stato creato 
esisste gi\`a
pu\`o essere costoso quindi
serve una struttura dati che permetta %di trovare ed 
estrarre (se c'\`e) il nodo in questione
nel minor tempo possibile. Per questo motivo si \`e scelto di usare una \emph{tabella hash}
%di tipo 
ad \emph{indirizzamento aperto} 
%per risolvere le collisioni.
in caso di collisioni.
La scelta %dell'indirizzamento
%aperto 
\`e dovuta al fatto che se si verifica una collisione possiamo utilizzare il \emph{doppio hashing}
%che prevede un'
con esplorazione %delle celle 
della tabella in modo diverso per ogni chiave, velocizzando
l'estrazione. Rispetto al metodo del concatenamento, inoltre non usa %ndo 
liste per memorizzare i nodi che collidono
%nella stessa cella, 
quindi
non si consumer\`a ulteriore memoria. % istanziando nuovi oggetti.
%Un altra 
Caratteristica fondamentale \`e che i nodi non vengono mai eliminati e quindi cancellati 
dalla tabella e questa \`e proprio la situazione ideale per questo tipo di tabella.

In fase di costruzione, ogni nodo `funzione' viene collegato tramite il campo `\campo{param}' ai nodi 
che sono suoi parametri e siccome sui parametri non ci sono operazione di unione si \`e 
scelto di usare un array visto che sappiamo esattamente l'ariet\`a di una funzione. 
Dualmente, ai nodi argomenti viene aggiunto il nodo funzione al
campo `\campo{ccpar}' che rappresenta i suoi padri. 
Qui per\`o %non \`e possibile 
l'uso dell'array
\`e improponibile
perch\'e questo campo %che 
%varia %sia durante la costruzione del grafo sia durante
%l'algoritmo di chiusura di congruenza
\`e soggetto a frequenti unioni, quindi si \`e
dovuti ricorrere ad un altra struttura dati diversa %(la scelta verr\`a specificata successivamente~[\ref{sec: algo cc}]).
come vedremo successivamente.
%A dir la verit\`a sono stati implementati due algoritmi
%Oltre a questi 
Poi \`e presente anche il campo `\campo{fn}'
che identifica il nome della funzione del nodo. 

Per gestire il fatto che un nodo deve essere atomo
o lista sono stati inseriti i booleani `\campo{atom}' e `\campo{cons}' che se messi a true indicano
rispettivamente se un nodo \`e atomo o !atom. Anche se sembra che siano rindondanti, perch\'e 
una situazione esclude l'altra, si sono usati due campi distinti perch\'e se sono messi entrambi a
false significa che non si sa di che tipo sia il nodo.
Questo ci permette di effettuare la prima ottimizzazione riconoscendo come insoddisfacibili le stringhe 
che contengono un elemento `cons' al posto di `atom' direttamente durante il parsing,
 per esempio car(cons(x,y)) oppure cons(cons(a,b),b). 
Poi il campo `\campo{find}' punta al 
rappresentante della classe di congruenza a cui appartiene il nodo (all'inizio ogni nodo appartiene
ad una classe differente, quindi punter\`a a se stesso), in questo modo ogni classe \`e rappresentata
come un albero e le varie classi come una foresta.
Per 
ottimizazione %controllare le disuguaglianze 
si \`e infine aggiunto il campo `\campo{diversi}' che contiene
tutti i nodi che non possono appartenere alla stessa classe del nodo e anche qui \`e importante scegliere
una buona struttura dati.

Assieme al grafo vengono poi costruite due code fifo che contengono coppie di nodi che sono in uguaglianza
o disuguaglianza tra loro. In questo modo si passa la coda delle uguaglianze all'algoritmo di chiusura
di congruenza e in caso questi restituisca ``soddisfacibile'', si verificano che anche le disuguaglianze 
siano rispettate.

\vspace{-1ex}
\section{Algoritmo di chiusura di congruenza}\label{sec: algo cc}
Per ogni uguaglianza rilevata nel parsing, si effettua il \emph{merge}. % come visto in classe.
Gi\`a in questa fase si possono fare delle ottimizzazioni. Cominciamo col fatto che invece di
passare i nodi al merge, si passano direttamente i loro \emph{rappresentanti} e solamente 
se questi sono diversi. Questo ci permette di non effettuare %gli stessi 
controlli rindondanti risparmiando le operazioni di \emph{find}(n1), \emph{find}(n2) e il loro confronto.
Infatti notiamo che anche l'algoritmo per trovare gli elementi congruenti
le fa gi\`a
prima
%su cui fare il 
del merge. % in quel momento. 
In pi\`u, nella \emph{union} quando si imposta un nodo come rappresentante dell'altro
creerebbe una catena di nodi fino ad arrivare al vero rappresentante; quindi 
trovare il rappresentate sarebbe dispendioso perch\'e bisognerebbe scorrerla tutta.
Il tempo richiesto verrebbe ridotto al minimo adottando l'accorgimento precedente.
Comunque si pu\`o sempre adottare la \emph{compressione dei cammini} che durante la
risalita della catena fa puntare il campo \campo{find} di ogni nodo incontrato al 
vero rappresentante.

Passiamo ad analizare il campo \campo{ccpar}. 
Qui occorre scegliere attentamente la struttura dati perch\'e i nodi in esso contenuti
vengono continuamente uniti nella \emph{union}. %A prima occhiata s
Si potrebbe decidere
di utilizzare una lista con puntatore al primo ed all'ultimo elemento %cos\`i si
%pu\`o unire in coda 
con unione in tempo costante, per\`o se andiamo a modificare cos\`i una lista
si incontra un problema quando chiamiamo la funzione \emph{espandiCongruenze} 
che analizza ogni possibile coppia tra i campi \campo{ccpar} per trovare qualche 
nuova congruenza su cui fare il merge. %da propagare sulle liste precedenti all'unione. 
% e poi viene controllata ogni
%ogni possibile coppia tra i nodi del campo prima che l'unione due 
%
Quindi nel `merge' si deve salvare il campo \campo{ccpar} dei due nodi dati in input.
Si sono pensate tante soluzioni, la pi\`u semplice \`e quella di salvare i nodi
contenuti in \campo{ccpar} in due array stando sicuri che li sarebbero rimasti inalterati
anche dopo l'unione. Questo per\`o ha un costo \emph{lineare} nel numero di nodi contenuti
nel campo.
Poi c'\`e da considerare i doppi. Attaccando semplicemente una lista in fondo all'altra
non ci permette di analizzare i nodi ripetuti e questo, oltre a consumare risorse,
ci aumenterebbe il numero delle coppie da analizzare.

A questo punto si sono sviluppate due diverse implementazioni, una per gestire
efficientemente i doppi e l'altra cercando di usufruire dell'unione in tempo lineare. % sarebbe molto interessante.
Per il primo scopo si \`e utilizzata una \emph{lista ordinata} cos\`i ogni volta che cerco di unire
un nodo che c'\`e gi\`a, lo trovo mentre cerco di inserirlo nella posizione ordinata.
Inoltre essendo entrambe gi\`a ordinate mi basta scorrerle entrambe solo una volta
risultando lineare nella somma del numero di nodi. In questo modo ci accorgiamo che
conviene costruire una nuova lista ordinata invece di unirla a quella del rappresentante.
Il tempo necessario \`e sempre lineare, per\`o
%Cos\`i 
non c'\`e bisogno di salvare le liste originali in un array.
Per il secondo scopo invece si \`e creata una struttura dati ad hoc chiamata \emph{DoppiaLista}
che non \`e altro che una lista doppiamente concatenata nella quale ci sono dei campi che
puntano alle due liste che la compongono cos\`i le liste sottostanti rimangono inalterate
e si pu\`o chiamare l'\emph{espandiCongruenze} su di esse. Evitando la copia negli array 
(come nel caso precedente) ed avendo un unione costante!.  Ma non \`e tutto oro quello
che luccica, infatti in fase implementativa si \`e notato che non gestendo gli elementi doppi
si ha uno spreco di memoria importante ed %le possibili combinazioni tra elementi gi\`a analizzati in
\emph{espandiCongruenze} considera combinazioni gi\`a incontrate in precedenza.
%precedenza crescono in maniera importante. 
Per ridurre questo fenomeno nell'\emph{espandiCongruenze} si \`e implementato un meccanismo 
di rimozione dei doppi, vanificando di fatto il vantaggio.
In entrambi i casi si \`e utilizzato l'ottimizzazione dell'\emph{unione per rango}
in modo da unire il pi\`u piccolo al pi\`u grande, cos\`i i cammini verso il rappresentanti
vengono allungati di uno solo sul numero minore di nodi.
Una caratteristica importante di entrambe le liste \`e l'estrazione in tempo costante
di ogni elemento se preso in ordine (proprio come accade in \emph{espandiCongruenze})
siccome la lista viene scansionata sempre tutta si utilizza un campo aggiuntivo \campo{index}
che ad ogni chiamata restituisce l'elemento puntato e si agiorna sul successivo.

Una volta completate le merge e le uguaglianze, ci restano da controllare le disuguaglianze
%trovate nel parsing 
cos\`i confrontiamo i rappresentanti dei nodi che devono essere
diversi. 
Per\`o cos\`i facendo ci accorgiamo solo alla fine di una disuguaglianza che 
non viene rispettata magari dopo la prima merge. Per questo %si \`e provveduto alla
%seguente ottimizzazione.
%In ogni nodo si \`e inserita una lista dei nodi %con i quali non deve condividere
%lo stesso rappresentante. Questa lista 
%che rappresenta i 
\`e utile il campo ``\campo{diversi}'' che viene 
mantenuto %aggiornata 
\emph{solo} nel rappresentante aggiungendo %i nodi contenuti
%nei nodi che si aggiungono. 
quelli
%nei campi dei nodi della sua stessa classe
di tutta la sua classe.
Questo campo viene controllato dopo ogni \emph{union}
cos\`i ci accorgeremmo tempestivamente di una eventuale insoddisfacibilit\`a.
Anche qui sono state fatte diverse prove. La lista ordinata \`e quella pi\`u
facile da gestrire e non rallenta troppo l'esecuzione. Si \`e provato anche una tabella
hash particolare (\emph{HashOpenPlusList}) nella quale vengono usate delle liste
per avere subito accessibili i nodi senza dover scandire tutte le celle della tabella.
%Nella fase di parsing viene riempita con i diversi una lista temporanea di supporto dell'hashtable, 
%e poi quando il grafo \`e completato si usa un metodo che crea l'effettiva tabella 
Questa classe crea tabelle
dalle dimensioni opportune (senza troppo spreco di memoria). 
%nella quale vengono copiati i nodi.
Questo %ulteriore passo 
rallenta un po' l'algoritmo per\`o permette di non dover trovare un compromesso
sulla dimensione iniziale della tabella evitando dimensioni %che in alcuni casi risulterebbe 
esagerate %ed in altri
%magari si \`e costretti a 
o frequenti rehash. 
Utilizzando questa hashtable
basta controllare che la classe di equivalenza di un nodo sia disgiunta dalla
tabella dei diversi.

Le diverse implementazioni comportano che le ``merge'' 
possono essere eseguite in ordine diverso
tra le due soluzioni, quindi si \`e analizzato il caso peggiore %per il rilevamento 
che si ha quando una formula \`e soddisfacibile perch\'e in entrambi i casi vengono
analizzate tutte.
Dai rilevamenti fatti risulta pi\`u efficiente la seconda versione con ``DoppiaLista'' costruita ad hoc.
La differenza non si nota tanto nell'esecuzione dell'algoritmo di chiusura di congruenza,
ma nella costruzione del grafo.
 

%, questo
%ha creato dei problemi, perch\'e prima che questi campi vengano letti nella funzione chiamata
%\emph{espandiCongruenze} 

\end{document}
